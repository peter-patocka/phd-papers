% additional options: [seceqn,secthm,crcready]
\documentclass[ais]{iosart2x}

%% Packages
\usepackage{dcolumn}
\usepackage[utf8]{inputenc}
\usepackage{hyperref}

%% Definitions
\newcolumntype{d}[1]{D{.}{.}{#1}}

%% Article Info
\firstpage{1}
\lastpage{1}
\volume{1}
\pubyear{2022}

\begin{document}
\begin{frontmatter}

%
%\pretitle{Pretitle}
\title{Ambient Intelligence: House predictive control based upon the user actions}

\runtitle{Ambient Intelligence: House predictive control based upon the user actions}
%\subtitle{Subtitle}

\begin{aug}
\author[A]{\inits{PP.}\fnms{Peter} \snm{Patočka}\ead[label=e1]{patocka.peter@gmail.com}}
\author[B]{\inits{VJ.}\fnms{Vladimír} \snm{Janoušek}\ead[label=e2]{janousek@fit.vutbr.cz }}
\address[A]{Department of Intelligent Systems, \orgname{Brno University of Technology}, \cny{Czech~Republic}}
\end{aug}

\begin{abstract}
Ambient Intelligence is in the real world represented by a multi-agent system where various intelligent agents
are taking actions to fulfill their goals and integrate them into common, more complex objectives.
Users usually interact with the system directly, but how to design the architecture of the system which
autonomously predicts users' needs?
Users could be owners of the system who have control over the mechanisms or simple visitors with passive
participation, but their presence is considered and taken into account either way.
One of the biggest challenges in the field is detection of human presence and human verification.
Multiple sorts of sensors should be used to determine the exact position and state of the user.
Ontologies help us to differentiate between users and categorize them.
Then data should be stored into the distributed, time series database by using stream processing systems,
where they are systematically extracted.
Big Data as the fuel for Artificial Intelligence analyzes users actions, tries to predict their next steps
and create automated rules or triggers to minimize user interactions with the systems.
In this paper I construct an overview of the system as such and explain a specific approach to predict
users' future steps by using formalism for modeling and analysis of discrete event systems (DEVS).
\end{abstract}

\begin{keyword}
\kwd{Ambient intelligence}
\kwd{artificial intelligence}
\kwd{multi-agent system}
\kwd{human presence detection}
\kwd{user action prediction}
\kwd{intelligent agents}
\kwd{internet of things}
\kwd{big data}
\kwd{discrete event system specification}
\end{keyword}
\end{frontmatter}

\section{Introduction}

Ambient intelligence (AmI) is intrinsically and thoroughly connected with artificial intelligence (AI).
Some even say that it is, in essence, AI in the environment~\cite{1}.

\begin{thebibliography}{0}

\bibitem{1} Matjaz Gams, Irene Yu-Hua Gu, Aki Härmä, Andrés Muñoz and Vincent Tam,
{Artificial intelligence and ambient intelligence},
{\textit{Journal of Ambient Intelligence and Smart Environments}} \textbf{11}(1) (2019), 71-86. ISSN 18761372. \bid{doi={10.3233/AIS-180508}}

\bibitem{2} M. Raval, S. Bhardwaj, A. Aravelli, J. Dofe, and H. Gohel,
{Smart energy optimization for massive IoT using artificial intelligence},
{\textit{Internet of Things}} \textbf{13}(1) (2021), p. 12. ISSN 25426605. \bid{doi={10.1016/j.iot.2020.100354}}

\bibitem{3} S. J. Russell and P. Norvig,
\textit{Artificial intelligence: a modern approach}, 3rd ed. Harlow: Pearson Education, 2014.

\bibitem{4} A. Pérez-vereda, D. Flores-martín, C. Canal, and J. M. Murillo,
{Towards Dynamically Programmable Devices Using Beacons},
in {\textit{Current Trends in Web Engineering}} (2018), vol. 11153, pp. 49-58. ISBN: 9783030030551. ISSN: 0302-9743. \bid{doi={10.1007/978-3-030-03056-8_5}}

\bibitem{5}  Streitz, Norbert,
{Beyond ‘smart-only’ cities: redefining the ‘smart-everything’ paradigm},
{\textit{Journal of ambient intelligence and humanized computing}} [online] \textbf{10}(2) (2012), 791-812. ISSN 1868-5137. \bid{doi={10.1007/s12652-018-0824-1}}

\bibitem{6} A. Kapoor,
\textit{Hands-on artificial intelligence for IoT: expert machine learning and deep learning techniques for developing smarter IoT systems}. Birmingham: Packt Publishing, 2019.

\end{thebibliography}


\end{document}
